\RequirePackage[l2tabu, orthodox]{nag}
\documentclass[]{memoir}

\usepackage{microtype}
\usepackage[dutch]{babel} 
\usepackage{titling}

%opening
\begin{document}


\begingroup% Scripts, T&H p 151
\centering
\vspace*{0.1\textheight}
{\Huge Het Access Management Systeem}\\[\baselineskip]
{\large\itshape Handleiding}\\[\baselineskip]
\vfill
\rule{0.4\textwidth}{0.4pt}\\[\baselineskip]
{\large\itshape Jeroen De Clerck}\par
\vspace*{0.1\textheight}
\endgroup

\clearpage
\tableofcontents

\chapter{Interface}
Om de gebruiker wegwijs te maken zullen we eerst in het kort alle elementen van het systeem overlopen. Deze zullen later in detail terugkeren waar nodig.

\section{Dashboard}
Het centraal overzicht. Bevat een historiek met het aantal personen toegevoegd aan het systeem per dag, plus het aantal verstuurde aanvraagformulieren (\textsl{Submitted Request Forms}) per dag. Dit overzicht kan manueel ververst worden door op het refresh-icoontje te klikken in de rechterbovenhoek. Exacte statistieken van een dag kunnen bekeken worden door met de muis over de grafiek te zweven.

\section{Persons}
Een overzicht van alle personen in het systeem. In de tabel kan men de naam, organisatie, categorie, telefoonnummer, aanwezigheid per dag, recht op voorwerpen, laatste aanpassing en status zien.

Op iedere persoon kan men een aantal acties uitvoeren, namelijk Toestaan (\textsl{Approve}), Afwijzen  (\textsl{Deny}), Aanpassen (\textsl{Edit}) en Verwijderen (\textsl{Delete}).

Men kan op meerdere personen tegelijk deze acties uitvoeren door ze aan te vinken, en vervolgens in de rechterbovenhoek een optie te selecteren uit het Groepacties  (\textsl{Group Actions}) selectiemenu.

Boven de tabel vindt men een filters en een zoekbalk, waarmee respectievelijk alleen de personen aanwezig op de geselecteerde datum getoond worden en respectievelijk met behulp van een tekstveld gezocht kan worden naar specifieke personen. Als laatste zijn er nog twee icoontjes, waarvan de eerste toestaat om een persoon aan te maken op basis van een excel-document, en waarvan het tweede toelaat om manueel een nieuwe persoon toe te voegen aan het systeem.

\section{Partners}

\section{Box Office}

\section{Rental}

\section{Reports}

\section{Configuration}
\subsection{Users}
\subsection{Roles}
\subsection{Categories}
\subsection{Items}
\subsection{Materials}
\subsection{Zones}
\subsection{General}
\subsection{Integrations}
\subsection{Fields}

\chapter{Gebruiksaanwijzingen}

\section{Box Office}
\subsection{Inchecken}
\subsection{Quick Add}

\section{Team Leads}
\subsection{Een Partner uitnodigen}

\end{document}
