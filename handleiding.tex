\RequirePackage[l2tabu, orthodox]{nag}
\documentclass[]{memoir}

\usepackage{microtype}
\usepackage[dutch]{babel} 
\usepackage[utf8]{inputenc}
\usepackage{titling}
\usepackage{graphicx}
\usepackage{float}

\graphicspath{{resources/}}
\textwidth=15cm
\setlength{\oddsidemargin}{12pt} 
\setlength{\evensidemargin}{12pt}

%opening
\begin{document}


\begingroup% Scripts, T&H p 151
\centering
\vspace*{0.1\textheight}
{\Huge Het Access Management Systeem}\\[\baselineskip]
{\large\itshape Handleiding}\\[\baselineskip]
\vfill
\rule{0.4\textwidth}{0.4pt}\\[\baselineskip]
{\large\itshape Jeroen De Clerck}\par
\vspace*{0.1\textheight}
\endgroup
\thispagestyle{empty}

\clearpage
\pagenumbering{roman}
\tableofcontents
\clearpage

\pagenumbering{arabic}
%%\chapter{Introductie}
%% wat uitleg over de handleiding zelf, 
%% een verduidelijking van het verschil tussen een Persoon (person), een Partner (Partner) en een Gebruiker (User),
%% een verduidelijking van het verschil tussen een Item en een Material


%% Noot aan schrijver: in de handleiding worden Persoon, Partner en Gebruiker geschreven met hoofdletter als het gaat over een element in de database van het systeem


\chapter{Interface} \label{Interface}
Om de gebruiker wegwijs te maken zullen we eerst in het kort alle elementen van het systeem overlopen. Deze zullen later in detail terugkeren waar nodig. Niet alle opties zijn voor alle gebruikers zichtbaar, dus als er bepaalde elementen onvindbaar zijn, is dat omdat de lezer er niet de correcte rechten voor heeft. Er wordt ten zeerste aangeraden om de onderdelen waar de lezer recht op heeft aandachtig te overlopen voor het lezen van de specifiekere gebruiksaanwijzingen.

\section{Dashboard} \label{Dashboard}
Het centraal overzicht. Bevat een historiek met het aantal personen toegevoegd aan het systeem per dag, plus het aantal verstuurde aanvraagformulieren (\textsl{Submitted Request Forms}) per dag. Dit overzicht kan manueel ververst worden door op het refresh-icoontje te klikken in de rechterbovenhoek. Exacte statistieken van een dag kunnen bekeken worden door met de muis over de grafiek te zweven.

\section{Persons} \label{Persons}
Een overzicht van alle Personen in het systeem. In de tabel kan men de naam, organisatie, categorie, telefoonnummer, aanwezigheid per dag, recht op voorwerpen (\textsl{Items}), laatste aanpassing en status zien. Bij sommige velden is er meer informatie beschikbaar door erover te zweven met de muis. Zo kunnen de meerdere categorien of dagen van een persoon bekeken worden door te zweven over het 'Various' veld in de respectievelijke kolom, of kan het specifieke Voorwerp waar een persoon recht op heeft bekeken worden door te zweven over het Voorwerp-icoontje.\\
Op iedere persoon kan men een aantal acties uitvoeren, namelijk:
\begin{itemize}
	\item Toestaan (\textsl{Approve}), waarmee een persoon toegevoegd via een form toegestaan wordt om in het systeem terecht te komen.
	\item Afwijzen  (\textsl{Deny}), waarmee een persoon toegevoegd via een form afgewezen wordt.
	\item Ticket versturen (\textsl{Send Ticket}), waarmee een mail verstuurd wordt naar het emailadres van de persoon met hun ticket en verdere informatie. Deze optie is enkel zichtbaar nadat de Persoon togestaan werd.
	\item Aanpassen (\textsl{Edit}), waarmee de Gebruiker naar het profielscherm van de persoon kan gaan, om daar eventueel velden aan te passen.
	\item Verwijderen (\textsl{Delete}), waarmee de persoon uit het systeem verwijderd wordt.
\end{itemize}

Men kan op meerdere personen tegelijk deze acties uitvoeren door ze aan te vinken, en vervolgens in de rechterbovenhoek een optie te selecteren uit het Groepacties  (\textsl{Group Actions}) selectiemenu.\\
Boven de tabel vindt men een filters en een zoekbalk, waarmee respectievelijk alleen de personen aanwezig op de geselecteerde datum getoond worden en respectievelijk met behulp van een tekstveld gezocht kan worden naar specifieke personen. Als laatste zijn er nog twee icoontjes, waarvan de eerste toestaat om een persoon aan te maken op basis van een excel-document, en waarvan het tweede toelaat om manueel een nieuwe persoon toe te voegen aan het systeem.\\


\section{Partners} \label{Partners}
Een overzicht van de Partners ingeschreven in het systeem. Bij het openen ziet de lezer enkel de partners die door hem ingeschreven zijn, maar men kan alle partners zien door linksboven op 'Alle Partners' (\textsl{All Partners}) te klikken. Terugkeren naar het beginscherm kan door op 'Mijn Partners' (\textsl{My Partners}) te klikken.\\
In de partnertabel ziet men de naam van de organisatie, de naam van de verantwoordelijke, of hij al dan niet ingelogd heeft, hoeveel van zijn tickets er verdeeld zijn, welk materiaal de Partner aanvraagt, de laatste aanpassing aan de Partner, de status en de Gebruiker die deze Partner aangemaakt heeft.\\
Op iedere Partner kan men een aantal acties uitvoeren, namelijk:
\begin{itemize}
	\item Toestaan (\textsl{Approve}), waarmee een Partner toegevoegd via een form toegestaan wordt om in het systeem terecht te komen.
	\item Afwijzen  (\textsl{Deny}), waarmee een Partner toegevoegd via een form  afgewezen wordt.
	\item ga naar Partner Service Center  (\textsl{Visit PSC}), waarmee de Gebruiker naar het Partner Service Center (PSC) van deze Partner gestuurd wordt. Deze link opent in een nieuw tabblad. Deze optie is enkel zichtbaar nadat de Persoon toegestaan werdt.
	\item verstuur Loginmail  (\textsl{Send Login}), waarmee een mail verstuurd wordt naar het emailadres van Partnerverantwoordelijke met een link naar hun PSC en verdere informatie. Deze optie is enkel zichtbaar nadat de Persoon toegestaan werdt.
	\item Aanpassen (\textsl{Edit}), waarmee de Gebruiker naar het profielscherm van de Partner kan gaan, om daar eventueel velden aan te passen.
	\item Verwijderen (\textsl{Delete}), waarmee de persoon uit het systeem verwijderd wordt.
\end{itemize}
Men kan op meerdere Partners tegelijk deze acties uitvoeren door ze aan te vinken, en vervolgens in de rechterbovenhoek een optie te selecteren uit het Groepacties  (\textsl{Group Actions}) selectiemenu.\\
Als laatste UI-element is er het plus-icoontje rechtsbovenaan de tabel die toestaat om een nieuwe partner aan te maken.

Het is belangrijk om het Partnersysteem niet te verwarren met het Personensysteem. Een Partner is een individu of organisatie, die het recht heeft om een gegeven aantal tickets uit te delen aan collegas. Zo kan bijvoorbeeld een sponsor een Partner worden, waardoor de sponsor een aantal vip-tickets toegewezen krijgt die uitgedeeld kunnen worden. \textsl{Het uitdelen van deze tickets is dan ook de verantwoordelijkheid van de partner.}

\subsection{Het Partner Service Center} \label{PSC}
Het Partner Service Systeem is dat wat de verantwoordelijke van een Partner te zien krijgt. Het bevat wat informatie over het systeem, contactgegevens van de Gebruiker die de Partner aan het systeem toegevoegd heeft.
De belangrijkste functie van deze pagina is de 'voeg persoon toe'-knop, die een partnerverantwoordelijke toelaat om mensen uit te nodigen. Als de Partner recht heeft op materiaal, is ook de knop 'Alles over voorzieningen' te zien, waarmee de partnerverantwoordelijke specifieke hoeveelheiden van het beschikbare materiaal kan reserveren.

\section{Box Office} \label{Box Office}
Het scherm voor tickets in te scannen en de bijhorende polsbandjes af te printen. De Gebruiker kan ofwel de barcode op het ticket inscannen of de Persoon opzoeken in het systeem.
Moest de Persoon niet in het systeem zitten, kan een Gebruiker met de correcte rechten een Persoon toevoegen via ofwel de \textsl{Quick Add} of de \textsl{Add Person} knop.

\section{Rental} \label{Rental}
Het overzicht voor Materials uit te lenen en terug te nemen. De Gebruiker kan op de grote IN/OUT knoppen klikken om respectievelijk een Material terug te nemen of uit te lenen. \\
Het Fastscan veld laat toe om het systeem zelf te laten opzoeken of het Materiaal al uitgeleend is. Als dat het geval is, zal het Materiaal automatisch ingeleverd worden. Zoniet zal het Materiaal automatisch uitgeleend worden.

\section{Reports} \label{Reports}
Geeft een overzicht van:
\begin{itemize}
	\item  hoeveel Personen van welke Categorie er op welke eventdagen toegang hebben.
	\item  hoeveel Items er op welke eventdagen gereserveerd zijn.
\end{itemize}

\section{Configuration} \label{Configuration}
Staat een Gebruiker toe om allerhande instellingen van het syteem aan te passen.
\subsection{Users}
Toont een tabel van Gebruikers, met respectievenlijk hun naam, emailadres, beschikbare UI-elementen en aanmaakdatum. 
Op iedere Gebruiker kan men een aantal acties uitvoeren, namelijk:
\begin{itemize}
	\item Aanpassen (\textsl{Edit}), waarmee men naar het profielscherm van de Gebruiker kan gaan, om daar eventueel velden aan te passen.
	\item Verwijderen (\textsl{Delete}), waarmee de Gebruiker uit het systeem verwijderd wordt.
\end{itemize}
Men kan op meerdere Gebruikers tegelijk deze acties uitvoeren door ze aan te vinken, en vervolgens in de rechterbovenhoek een optie te selecteren uit het Groepacties  (\textsl{Group Actions}) selectiemenu.\\
Verder kan men met behulp van de zoekbalk zoeken naar een specifieke gebruiker, en kan met met behulp van het plus-icoontje in de rechterbovenhoek een nieuwe gebruiker aanmaken.

\subsection{Roles} \label{Roles}
Staat toe om een aantal voorgedefinieerde Rollen aan te maken die specifieke vergunningen in het systeem hebben, en die vervolgens toegewezen kunnen worden aan een Gebruiker.\\
langs de linkerkant staan de geconfigureerde Rollen, en langs de rechterkant kan men een nieuwe Rol aanmaken. 
Specifieke Rollen kunnen verwijderd (\textsl{Delete}) of aangepast (\textsl{Edit}) worden met behulp van het actiemenu (\textsl{Action}).
Men kan meerdere Rollen tegelijk verwijderen door ze aan te vinken, en vervolgens in de rechterbovenhoek 'verwijderen' te selecteren uit het Groepacties  (\textsl{Group Actions}) selectiemenu.\\

\subsection{Categories} \label{Categories}
Staat toe om een aantal voorgedefinieerde Categorieën aan te maken die toegewezen kunnen worden aan een Persoon.
In de tabel ziet men een opsomming van de huidige Categorien, met hun naam, de naam die voor deze categorie gebruikt wordt in het Partner Service Centrum, en of al dan niet hun mailtekst, tickettekst en procedure ingevuld zijn in het systeem.
Op iedere Categorie kan men naast Aanpassen (\textsl{Edit}) en Verwijderen (\textsl{Delete}) ook nog een aantal acties uitvoeren, namelijk:
\begin{itemize}
	\item Aanvraagformulier bekijken (\textsl{Visit Request Form}), waarmee de gebruiker het aanvraagformulier voor een ticket van die categorie te zien krijgt. Deze link opent in een nieuw tabblad.
	\item Exporteer (\textsl{Export}), waarmee informatie over de gebruikers van die categorie in de vorm van een Excel-bestand gedownload kan worden.
\end{itemize}
Men kan meerdere Categorieën tegelijk verwijderen door ze aan te vinken, en vervolgens in de rechterbovenhoek de `verwijderen' optie te selecteren uit het Groepacties  (\textsl{Group Actions}) selectiemenu.

\subsection{Items} \label{Items}
Een overzicht van de Voorwerpen (\textsl{Items}) in het systeem. In de tabel ziet men de naam, onder welke verzamelnaam het voorwerp zich bevindt, of het aan een Persoon of een Partner uitgeleend moet worden, een kolom met icoontjes die extra informatie toont en de aanmaakdatum van het Voorwerp.
De kolom met icoontjes geeft een snel overzicht of een Voorwerp zichtbaar is in het Verslagmenu (\textsl{Reports}), of het een extra ticket nodig heeft, en of het Voorwerp enkel op specifieke dagen beschikbaar is. Specifieke Voorwerpen kunnen verwijderd (\textsl{Delete}) of aangepast (\textsl{Edit}) worden met behulp van het actiemenu (\textsl{Action}).

Onderaan de pagina zijn ook de Voorwerpveld- en Voorwerpgroeptabellen zichtbaar. Dit zijn onderdelen van een Voorwerpbeschrijving en kunnen op dezelfde manier als een Voorwerp aangemaakt, aangepast of verwijderd worden.

\subsection{Materials} \label{Materials}
Het is belangrijk om een Voorwerp niet te verwarren met Materiaal. Een Voorwerp heeft een stock, en er kunnen dus 1 of meerdere zelfde Voorwerpen uitgeleend worden aan verschillend Personen of Partners. Een Voorwerp kan uitleenbaar (\textsl{Rentable}) gemaakt worden, waarna ieder apart instantie van dit Voorwerp een barcode moet toegewezen worden in het Materiaalmenu, zodat ze via het Uitleenmenu (\textsl{Rental}) uitgeleend kan worden.

Links op het scherm ziet men een overzicht van alle Materialen in het systeem met hun naam, barcode en Voorwerptype. Men kan deze aanpassen of verwijderen met behulp van het actieselectieveld (\textsl{Action})
Rechts op het scherm kan de Gebruiker materiaal toevoegen door het Voorwerptype te selecteren, de naam in te geven en de barcode te scannen.
\subsection{Zones} \label{Zones}
Links staat een overzicht van alle Zones op het terrein, rechts een klein menuutje waar nieuwe zones mee aangemaak kunen worden. Hier kunnen weer Zones aangemaakt of aangepast worden, per stuk of in groep, met behulp van de Actie- en Groepactieknoppen.

\subsection{General} \label{General}
Bevat een hoop instellingen met betrekking tot het festival zelf (o.a. tickettemplates, tekstjes voor mails en het Partner Service Systeem). Meer info hierover is te vinden onder Sectie \ref{SetSystem}.

\subsection{Integrations} \label{Integrations}
Hier kan men de ingestelde Integraties (zoals een planningstool) bezichtigen.

\subsection{Fields} \label{Fields}
Hier kan men extra velden maken die dan ingevuld kunnen worden bij het aanmaken of aanpassen van een Person.

\chapter{Gebruiksaanwijzingen} \label{Instructions}

\section{Admin} \label{AdminInstr}
\subsection{Het systeem instellen.} \label{SetSystem}
Ga naar het Configuratie$\rightarrow$Algemeen menu (zie \ref{General}). Hier ziet men de algemene instellingen van het systeem. Ieder veld met een sterretje naast is een veld dat verplicht ingevuld moet worden.
\subsubsection{General}
Hier staat er basisinformatie van het evenement.
\subsubsection{Mails}
Hier wordt de ticketmail ingesteld. Men kan hier respectievelijk de naam van de verzender, het adres van de verzender, het onderwerp van de mail, de tekst van de mail, de voetnote voor het onleesbaar zijn van de mail en de voetnote voor een link naar een online versie van de mail aanpassen.
\subsubsection{Tickets}
Hier wordt de achtergrondafbeelding van de printbare tickets ingesteld. De bedoeling is dat men op basis van het gegeven template een nederlandstalige en engelstalige versie van en ticket maakt en deze upload in het systeem. Eens dat deze afbeeldingen zich in het systeem bevinden kunnen ze opnieuw bekeken worden door op de bestandsnaam te klikken.
\subsubsection{PSC}
\subsection{Een Gebruiker toevoegen} \label{AddUser}
\subsection{Een Rol toevoegen} \label{SetRole}
\subsection{Een Categorie toevoegen} \label{SetCategory}
\subsection{Een Voorwerp toevoegen} \label{SetItem}
Voor een Voorwerp toe te voegen aan het systeem, ga naar het Items (\ref{Items}) menu. \\
\begin{figure}[H]
	\includegraphics[width=\linewidth]{addItem_left}
	\caption{een Voorwerp aanmaken}
\end{figure}
Vul de naam van een enkelvoudig item en het meervoud van deze naam in.
Vervolgens kan men optioneel nog het maximum aantal van dit Voorwerp invullen onder Maximum,
Selecteer in het Target selectiemenu of dit item voor Partners of Personen is. \\
Vervolgens zijn er een aantal aantikbare opties.
\begin{itemize}
	\item \textsl{Rentable} zorgt dat dit Voorwerp geïndexeerd kan worden als Materiaal en bijgevolg uitgeleend kan worden in het Rentalmenu (\ref{Rental}), 
	\item \textsl{Show in grid} zorgt dat het item zichtbaar is in het Reportsmenu (\ref{Reports}), 
	\item \textsl{Date dependent} betekent dat het Voorwerp per dag gereserveerd moet worden.
	\item \textsl{Extra Ticket} zorgt dat bij het reserveren van het Voorwerp een extra ticket afgeprint moet worden (denk aan Parkeertickets). Als dit laatste aangevinkt wordt moet er ook een tweetalige tickettekst opgegeven worden.
\end{itemize} 
\begin{figure}[H]
	\includegraphics[width=\linewidth]{addItem_right}
	\caption{een Voorwerp aanmaken}
	\label{fig:itemfield}
\end{figure}
Selecteer in het itemgroup selectiemenu onder welke itemgroup dit voorwerp valt. 
\begin{figure}[H]
	\includegraphics[width=\linewidth]{itemfield}
	\caption{een itemfield aanmaken}
\end{figure}
Voeg onder Item-fields toe welke welk soort extra informatie de Partner of Persoon moet opgeven bij het reserveren. Een bestaand Itemfield kan opgezocht worden door de naam in te tikken, een nieuw Itemfield aanmaken kan men door op het plus-icoontje te klikken en de relevante informatie in te vullen. In afbeelding \ref{fig:itemfield} wordt een itemfield met de naam 'Dit is een itemfield' aangemaakt, waarvan verplicht een checkbox aangetikt moet worden.
\begin{figure}[H]
	\includegraphics[width=\linewidth]{itemfield_result}
	\caption{Wat er gebeurt als een Persoon een Voorwerp met voorgaand Itemfield aanvraagt.}
\end{figure}
Ten laatste kan men een .pdf uploaden die samen met het ticket van de Persoon opgestuurd zal worden als deze Persoon dit item gereserveerd heeft. Mogelijke toepassingen zijn een grondplan bij het reserveren van locatie-afhankelijke Voorwerpen.
\subsection{Materiaal toevoegen} \label{SetMaterial}
Voor een Materiaal toe te voegen aan het systeem, ga naar het Materials (\ref{Materials}) menu. \\
\begin{figure}[H]
	\includegraphics[width=\linewidth]{addMaterial}
	\caption{Materiaal aanmaken}
\end{figure}
Als het gaat om een Voorwerp dat in het systeem zit, kan men dit selecteren in het item selectiemenu.
Vervolgens moet men een unieke identificerende naam invoeren (zoals itemnaam\_1) plus de barcode van dit materiaal.
Als deze velden ingevuld zijn kan met op de Saveknop klikken om het Materiaal toe te voegen aan het systeem.

\subsection{Een Zone toevoegen} \label{SetZone}
In het Zonemenu kan er rechts een nieuwe zone aangemaakt worden. Deze heeft een naam en een symbool.
\subsection{Integraties beheren} \label{SetIntegration}
Het integratiemenu wordt gebruikt om integraties met het access-management systeem te beheren. De enige integraties huidig beschikbaar zijn Ticketmatic, en de integratie met de planningstool.
\subsection{Een Veld toevoegen} \label{SetField}
Een veld toevoegen gebeurt in het ``Fields'' menu aan de rechterkant.

\section{Box Office} \label{BoxOfficeInstr}
\subsection{Inchecken} \label{BOCheckIn}
\subsection{Quick Add} \label{BOQuickAdd}

\section{Team Leads} \label{TeamLeadInstr}
\subsection{Een Partner uitnodigen} \label{AddPartner}
Voor een Parter toe te voegen aan het systeem, ga naar het Partners (\ref{Partners}) menu. Rechtsbovenaan kan u op de plus-knop klikken om naar de aanmaakpagina te gaan.
\begin{figure}[H]
	\includegraphics[width=\linewidth]{addPartner_general}
	\caption{algemene info van een Partner instellen}
\end{figure}
Hier kan men de voornaam, naam, mailadres, telefoonnummer, taal en organisatienaam van een Partner invullen. Verder is er nog aan aantikbaar veld om deze verantwoordelijke een link te sturen naar het Partner Service Center, waar hij/zij materiaal kan reservenen en andere mensen kan uitnodigen (deze zal voor het overgrote deel van de Partners aangevinkt moeten worden). Het is belangrijk dat men realiseert dat dit de informatie is van de Partnerverantwoordelijke, die verantwoordelijk is voor het verdelen van de uitnodigingen die deze partner toegewezen zijn.
\begin{figure}[H]
	\includegraphics[width=\linewidth]{addPartner_date}
	\caption{aantal uitnodigingen van een Partner instellen}
	\label{img:Partnerdate}
\end{figure}
Hier kan men uitnodigingen aan een Partner toewijzen. De Categorie van uitgenodigden kan aangepast worden in het Category selectiemenu. Er kan per dag ingevuld worden hoeveel tickets van deze Categorie uitgedeeld kunnen worden. \\  Om Voorwerpen toe te wijzen aan deze Categorie van personen, kan men bij het \textsl{Action} selectiemenu klikken op \textsl{Add items}, en vervolgens het gewenste Voorwerp uit de submenus selecteren. Via deze weg kunnen enkel Voorwerpen met Personen als doel toegewezen worden. Voorwerpen met Partners als doel toewijzen wordt even verderop afgehandeld. \\
Om meer dan 1 Categorie aan tickets aan een Partner te geven, kan men op het plus-icoontje in de rechterbovenhoek klikken, die dan op dezelfde manier ingevuld kan worden. 
Op figuur \ref{img:Partnerdate} heeft de  partner uitnodigingen voor 1 VIP op 21 september, 5 VIPs op de festivaldag (waarvan er 2 een avondmaal krijgen) en 1 fotograaf op de festivaldag.
\begin{figure}[H]
	\includegraphics[width=\linewidth]{addPartner_utilities}
	\caption{Voorwerpen toewijzen aan een Partner}
\end{figure}
In het laaste onderdeel kunnen Voorwerpen toegewezen worden aan de Partner zelf. De Voorwerpen zijn gesorteerd per icoontje, en ze kunnen toegewezen worden door we aan te vinken op de dag die van toepassing is. Eens dat alles ingesteld is, kan er een overzicht van alle toegewezen Voorwerpen gezien worden op de \textsl{Assigned Utilities} tab. 
\begin{figure}[H]
	\includegraphics[width=\linewidth]{addPartner_utilities_main}
	\caption{overzicht van de beschikbare Voorwerpen toegewezen aan een partner}
\end{figure}

\section{Partners} \label{PartnerInstr}
\subsection{Utilities aanvragen.} \label{RequestMaterial}

\end{document}
