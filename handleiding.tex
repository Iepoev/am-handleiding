\RequirePackage[l2tabu, orthodox]{nag}
\documentclass[]{memoir}

\usepackage{microtype}
\usepackage[dutch]{babel} 
\usepackage{titling}

%opening
\begin{document}


\begingroup% Scripts, T&H p 151
\centering
\vspace*{0.1\textheight}
{\Huge Het Access Management Systeem}\\[\baselineskip]
{\large\itshape Handleiding}\\[\baselineskip]
\vfill
\rule{0.4\textwidth}{0.4pt}\\[\baselineskip]
{\large\itshape Jeroen De Clerck}\par
\vspace*{0.1\textheight}
\endgroup
\thispagestyle{empty}

\clearpage
\pagenumbering{roman}
\tableofcontents
\clearpage

\pagenumbering{arabic}
\chapter{Interface}
Om de gebruiker wegwijs te maken zullen we eerst in het kort alle elementen van het systeem overlopen. Deze zullen later in detail terugkeren waar nodig. Niet alle opties zijn voor alle gebruikers zichtbaar, dus als er bepaalde elementen onvindbaar zijn, is dat omdat de lezer er niet de correcte rechten voor heeft. Er wordt ten zeerste aangeraden om de onderdelen waar de lezer recht op heeft aandachtig te overlopen voor het lezen van de specifiekere gebruiksaanwijzingen,

\section{Dashboard}
Het centraal overzicht. Bevat een historiek met het aantal personen toegevoegd aan het systeem per dag, plus het aantal verstuurde aanvraagformulieren (\textsl{Submitted Request Forms}) per dag. Dit overzicht kan manueel ververst worden door op het refresh-icoontje te klikken in de rechterbovenhoek. Exacte statistieken van een dag kunnen bekeken worden door met de muis over de grafiek te zweven.

\section{Persons}
Een overzicht van alle Personen in het systeem. In de tabel kan men de naam, organisatie, categorie, telefoonnummer, aanwezigheid per dag, recht op voorwerpen (\textsl{Items}), laatste aanpassing en status zien. Bij sommige velden is er meer informatie beschikbaar door erover te zweven met de muis. Zo kunnen de meerdere categorien of dagen van een persoon bekeken worden door te zweven over het 'Various' veld in de respectievelijke kolom, of kan het specifieke Voorwerp waar een persoon recht op heeft bekeken worden door te zweven over het Voorwerp-icoontje.\\
Op iedere persoon kan men een aantal acties uitvoeren, namelijk:
\begin{itemize}
	\item Toestaan (\textsl{Approve}), waarmee een toegevoegde persoon toegestaan wordt om in het systeem terecht te komen.
	\item Afwijzen  (\textsl{Deny}), waarmee een toegevoegde persoon afgewezen wordt.
	\item Ticket versturen (\textsl{Send Ticket}), waarmee een mail verstuurd wordt naar het emailadres van de persoon met hun ticket en verdere informatie. Deze optie is enkel zichtbaar nadat de Persoon togestaan werdt.
	\item Aanpassen (\textsl{Edit}), waarmee de Gebruiker naar het profielscherm van de persoon kan gaan, om daar eventueel velden aan te passen.
	\item Verwijderen (\textsl{Delete}), waarmee de persoon uit het systeem verwijderd wordt.
\end{itemize}

 
Men kan op meerdere personen tegelijk deze acties uitvoeren door ze aan te vinken, en vervolgens in de rechterbovenhoek een optie te selecteren uit het Groepacties  (\textsl{Group Actions}) selectiemenu.\\
Boven de tabel vindt men een filters en een zoekbalk, waarmee respectievelijk alleen de personen aanwezig op de geselecteerde datum getoond worden en respectievelijk met behulp van een tekstveld gezocht kan worden naar specifieke personen. Als laatste zijn er nog twee icoontjes, waarvan de eerste toestaat om een persoon aan te maken op basis van een excel-document, en waarvan het tweede toelaat om manueel een nieuwe persoon toe te voegen aan het systeem.\\
Om meer informatie te zien 

\section{Partners}
Een overzicht van de Partners ingeschreven in het systeem. Bij het openen ziet de lezer enkel de partners die door hem ingeschreven zijn, maar men kan alle partners zien door linksboven op 'Alle Partners' (\textsl{All Partners}) te klikken. Terugkeren naar het beginscherm kan door op 'Mijn Partners' (\textsl{My Partners}) te klikken.\\
In de partnertabel ziet men de naam van de organisatie, de naam van de verantwoordelijke, of hij al dan niet ingelogd heeft, hoeveel van zijn tickets er verdeeld zijn, \textbf{WAT ZIJN UTILITIES??????}, de laatste aanpassing aan de partner, de status en de Gebruiker die deze Partner aangemaakt heeft.\\
Op iedere Partner kan men een aantal acties uitvoeren, namelijk een link naar hun Partner Service Systeem  (\textsl{Visit PSC}), Een mail sturen met een link om in te loggen  (\textsl{Send Login}), Aanpassen (\textsl{Edit}) en Verwijderen (\textsl{Delete}).
Men kan op meerdere Partners tegelijk deze acties uitvoeren door ze aan te vinken, en vervolgens in de rechterbovenhoek een optie te selecteren uit het Groepacties  (\textsl{Group Actions}) selectiemenu.\\

Het is belangrijk om het Partnersysteem niet te verwarren met het Personensysteem. Een Partner is een individu of organisatie, die het recht heeft om een gegeven aantal tickets uit te delen aan collegas. Zo kan bijvoorbeeld een sponser een Partner worden, waardoor de sponser een aantal vip-tickets toegewezen krijgt die uitgedeeld kunnen worden. \textsl{Het uitdelen van deze tickets is dan ook de verantwoordelijkheid van de partner.}

\subsection{Het Partner Service Center}


\section{Box Office}

\section{Rental}

\section{Reports}

\section{Configuration}
\subsection{Users}
\subsection{Roles}
\subsection{Categories}
\subsection{Items}
\subsection{Materials}
\subsection{Zones}
\subsection{General}
\subsection{Integrations}
\subsection{Fields}

\chapter{Gebruiksaanwijzingen}

\section{Box Office}
\subsection{Inchecken}
\subsection{Quick Add}

\section{Team Leads}
\subsection{Een Partner uitnodigen}

\end{document}
